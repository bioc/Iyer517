
%
% NOTE -- ONLY EDIT Biobase.Rnw!!!
% Biobase.tex file will get overwritten.
%
%\VignetteIndexEntry{Iyer517}
%\VignetteDepends{Iyer517, Biobase}
%\VignetteKeywords{Expression Analysis, time course}
%\VignettePackage{Iyer517}


%
% NOTE -- ONLY EDIT THE .Rnw FILE!!!  The .tex file is
% likely to be overwritten.
%
\documentclass[12pt]{article}

\usepackage{amsmath,pstricks}
\usepackage[authoryear,round]{natbib}
\usepackage{hyperref}


\textwidth=6.2in
\textheight=8.5in
%\parskip=.3cm
\oddsidemargin=.1in
\evensidemargin=.1in
\headheight=-.3in

\newcommand{\scscst}{\scriptscriptstyle}
\newcommand{\scst}{\scriptstyle}

\bibliographystyle{plainnat} 
 
\usepackage{/home/stvjc/R-devel/share/texmf/Sweave}
\begin{document}

\title{An {\tt exprSet} for the Iyer genomic time series database}
\author{VJ Carey, {\tt stvjc@channing.harvard.edu}}
\maketitle

\section{Overview}
Iyer, Eisen et al (Science 1999, v283 83-87)
report a cDNA-chip based experiment to illustrate
the transcriptional response of fibroblasts to serum.
The original data are archived in full at 
\url{genome-www.stanford.edu/serum/data.html}.
This package provides access to a subset of the
data leading to Figure 2 of their paper.
It would be worthwhile to provide high-level objects
representing the entire dataset, and this will be
taken up in the future.

\section{The {\tt Iyer517} {\tt exprSet}}
To get access to the data, install the
{\it Iyer517} package and then attach it:
\begin{Schunk}
\begin{Sinput}
> library(Iyer517)
\end{Sinput}
\begin{Soutput}
Loading required package: Biobase
\end{Soutput}
\end{Schunk}
A summary of the key dataset is:
\begin{Schunk}
\begin{Sinput}
> show(Iyer517)
\end{Sinput}
\begin{Soutput}
Instance of ExpressionSet 

assayData
  Storage mode: lockedEnvironment 
  featureNames: W95909, AA045003, AA044605, ..., W90037, W88650 (517 total)
  Dimensions:
        exprs
Rows      517
Samples    19

phenoData
  sampleNames: E0HR, E15MIN, E30MIN, ..., E0HRC, EUNSYNC (19 total)
  varLabels:
    time.hrs: time, NA=Unsync
    cycloheximide: cycloheximide present

Experiment data
  Experimenter name: Iyer et al. 
  Laboratory: Department of Biochemistry, Stanford University School of Medicine, Stanford CA 94305, USA. 
  Contact information:  
  Title: The transcriptional program in the response of human fibroblasts to serum. 
  URL:  
  PMIDs: 9872747 

  Abstract: A 86 word abstract is available. Use 'abstract' method.

Annotation [1] ""
\end{Soutput}
\end{Schunk}
The first few expression records are:
\begin{Schunk}
\begin{Sinput}
> exprs(Iyer517)[1:4, 1:6]
\end{Sinput}
\begin{Soutput}
         E0HR E15MIN E30MIN E1HR E2HR E4HR
W95909      1   0.72   0.10 0.57 1.08 0.66
AA045003    1   1.58   1.05 1.15 1.22 0.54
AA044605    1   1.10   0.97 1.00 0.90 0.67
W88572      1   0.97   1.00 0.85 0.84 0.72
\end{Soutput}
\end{Schunk}
Note that columns 1 to 13 correspond to sampling times
in the absence of cycloheximide (an inhibitor of protein synthesis)
and columns 14 to 19 correspond to sampling times
in the presence of cycloheximide.  The tags {\tt UNSYN}
and {\tt UNSYNC} are sampling from cells in exponential
replication.

\section{Replication of some findings}

\subsection{Figure 2}
To reproduce Figure 2 we need a color scheme and
some transformations.  The following seems
to do a reasonable job:

\begin{Schunk}
\begin{Sinput}
> chg <- seq(0.1, 8, 0.01)
> mycol <- rgb(chg/8, 1 - chg/8, 0)
> CEX <- exprs(Iyer517)
> CEX[CEX > 8] <- 8
\end{Sinput}
\end{Schunk}
\begin{center}
\begin{Schunk}
\begin{Sinput}
> image(t(log10(CEX[517:1, 1:13])), col = mycol, xlim = c(0, 3), 
+     axes = FALSE, xlab = "Hours post exposure to serum")
> axis(1, at = (1:13)/13, lab = c("0", ".25", ".5", "1", "2", "4", 
+     "6", "8", "12", "16", "20", "24", "u"), cex = 0.3)
\end{Sinput}
\end{Schunk}
\includegraphics{Iyer517-005}
\end{center}

However, the time 0 column of Figure 2 in the paper
shows some variability.  This is hard to square with the
caption indicating that the data depicted are ratios relative
to time 0.

\subsection{The mean within-cluster trajectories}
To orient it seems we need clusters contiguous to
the boundaries of the image matrix, because there are gaps
of unspecified length between many of the clusters.

\begin{Schunk}
\begin{Sinput}
> par(mfrow = c(2, 2))
> plot(apply((CEX[1:100, 1:13]), 2, mean), main = "Cluster A", 
+     log = "y", ylab = "fold change", xlab = "index in timing sequence")
> plot(apply((CEX[101:242, 1:13]), 2, mean), main = "Cluster B", 
+     log = "y", ylab = "fold change", xlab = "index in timing sequence")
> plot(apply((CEX[483:499, 1:13]), 2, mean), main = "Cluster I", 
+     log = "y", ylab = "fold change", xlab = "index in timing sequence")
> plot(apply((CEX[500:517, 1:13]), 2, mean), main = "Cluster J", 
+     log = "y", ylab = "fold change", xlab = "index in timing sequence")
\end{Sinput}
\end{Schunk}
\includegraphics{Iyer517-006}

The trajectories are very similar to those reported in the paper.
\section{Extended annotation}
An effort has been made to incorporate GO tags into this data resource.
\begin{Schunk}
\begin{Sinput}
> data(IyerAnnotated)
> print(IyerAnnotated[1:5, ])
\end{Sinput}
\begin{Soutput}
  Iclust       GB seqno locusid        GO1        GO2  GO3  GO4  GO5
1      N   W95909     1   80298       <NA>       <NA> <NA> <NA> <NA>
2      A AA045003     2    6414 GO:0008430       <NA> <NA> <NA> <NA>
3      A AA044605     3    5300 GO:0003755 GO:0005515 <NA> <NA> <NA>
4      A   W88572     4    2037 GO:0005200       <NA> <NA> <NA> <NA>
5      A AA029909     5   51747       <NA>       <NA> <NA> <NA> <NA>
\end{Soutput}
\end{Schunk}
At the time of construction, at most 5 GO tags had been
associated with any probes in the dataset, and a large number
of probes lacked both Locus Link and GO tags.
\end{document}
